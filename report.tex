\PassOptionsToPackage{unicode=true}{hyperref} % options for packages loaded elsewhere
\PassOptionsToPackage{hyphens}{url}
%
\documentclass[]{article}
\usepackage{bm}
\usepackage{lmodern}
\usepackage{amssymb,amsmath}
\usepackage{ifxetex,ifluatex}
\usepackage{fixltx2e} % provides \textsubscript
\ifnum 0\ifxetex 1\fi\ifluatex 1\fi=0 % if pdftex
  \usepackage[T1]{fontenc}
  \usepackage[utf8]{inputenc}
  \usepackage{textcomp} % provides euro and other symbols
\else % if luatex or xelatex
  \usepackage{unicode-math}
  \defaultfontfeatures{Ligatures=TeX,Scale=MatchLowercase}
\fi
% use upquote if available, for straight quotes in verbatim environments
\IfFileExists{upquote.sty}{\usepackage{upquote}}{}
% use microtype if available
\IfFileExists{microtype.sty}{%
\usepackage[]{microtype}
\UseMicrotypeSet[protrusion]{basicmath} % disable protrusion for tt fonts
}{}
\IfFileExists{parskip.sty}{%
\usepackage{parskip}
}{% else
\setlength{\parindent}{0pt}
\setlength{\parskip}{6pt plus 2pt minus 1pt}
}
\usepackage{hyperref}
\hypersetup{
            pdfborder={0 0 0},
            breaklinks=true}
\urlstyle{same}  % don't use monospace font for urls
\usepackage[normalem]{ulem}
% avoid problems with \sout in headers with hyperref:
\pdfstringdefDisableCommands{\renewcommand{\sout}{}}
\setlength{\emergencystretch}{3em}  % prevent overfull lines
\providecommand{\tightlist}{%
  \setlength{\itemsep}{0pt}\setlength{\parskip}{0pt}}
\setcounter{secnumdepth}{0}
% Redefines (sub)paragraphs to behave more like sections
\ifx\paragraph\undefined\else
\let\oldparagraph\paragraph
\renewcommand{\paragraph}[1]{\oldparagraph{#1}\mbox{}}
\fi
\ifx\subparagraph\undefined\else
\let\oldsubparagraph\subparagraph
\renewcommand{\subparagraph}[1]{\oldsubparagraph{#1}\mbox{}}
\fi

% set default figure placement to htbp
\makeatletter
\def\fps@figure{htbp}
\makeatother


\title{Report on hiccups at recreating the Automatica paper}
\author{Drew Allen McNeely}

\begin{document}

\maketitle

In this miniature report, I will walk through my code along with a bit
of math from the Automatica paper to display where my issue is.

Before I go into that, I'll briefly go over the overarching problem I'm
having with Diciplined Convex Programming and the curvature/sign of the
sub-expressions within the performance index.

The matrices that are formed via affine functions of the desision
variable, ie. \(\mathcal{X}_0(\Psi)\), \(\mathcal{X}_w(\Psi)\),
\(\mathcal{U}_0(\Psi)\), \(\mathcal{U}_w(\Psi)\) get multiplied by their
respective transposes, whose trace is then taken in the cost function.
While \(\mathrm{tr}(AA^T)\) is a convex function in \(A\), it does not
follow the rules of Disciplined Convex Programming as laid out by CVXPY
because the sign of \(AA^T\) is unknown. From what I've read online, a
way to get around this is to reformulate the cost in terms of a
Kronecker product of two matrices. I am currently working on getting
this to work and using existing documentation as a guide. I was hoping
to get feedback on my thought process and formulation so that I may move
forward. I know that the Kronecker product is an alternative way to
express the tensor product as a matrix instead of a rank 4 covariant
tensor, but I fail to see how I can turn my reformulation into a
Kronecker product. I have my doubts that this process would actually
change my results. The matrix \(\mathrm{tr}(AA^T)\) itself has positive
sign and curvature, but CVXPY has no way of knowing what its sign is.
Currently, I am using the Boyd paper as a refrence point in order to get
my code functioning and eliminate the error that I'm having with
sign/curvature.

Below, I'm going to document what I've done and the issues that I am
currently having. Please let me know of any comments or suggestions you
have of how to make this work.

\section{Rederivation of the mathematics}

Here, I will rederive the math in this paper to make it just a bit easier to follow.
I'll compactify some arguments with the intent of making it as easy as possible to reformulate in a way that is accepted by CVX.

We start with equations (20a) through (21c) in the Automatica paper

\newcommand{\state}{\bm{x}}
\newcommand{\Xw}{\bm{\mathcal{X}_w}}
\newcommand{\Xo}{\bm{\mathcal{X}}_0}

\newcommand{\control}{\bm{u}}
\newcommand{\Uw}{\bm{\mathcal{U}_w}}
\newcommand{\Uo}{\bm{\mathcal{U}}_0}

\newcommand{\noise}{\bm{w}}
\newcommand{\init}{\bm{x}_0}

\begin{align}
	\state &= \Xw\noise + \Xo\init \\
	\control &= \Uw\noise + \Uo\init
\end{align}

with

\begin{align}
	\Xw(\Psi) &= (I + H\Psi)G \\
	\Xw(\Psi) &= (I + H\Psi) \\
	\Uw(\Psi) &= \Psi G \\
	\Uw(\Psi) &= \Psi
\end{align}

\newcommand{\statecontrol}{\bm{z}}
\newcommand{\zo}{\bm{z}_0}

In order to compactify these equations, we can combine the state space and the control space into one large space and define $\statecontrol = [\state^T \control^T]^T$.
We can similarly do so for the noise and propagated initial state: $\zo = [\noise^T \init^T]^T$.
\newcommand{\mapper}{\mathcal{M}}
Thus, we get $\statecontrol = \mapper \zo$ where

\begin{equation}
\begin{aligned}
	\mapper(\Psi) &=
	\begin{bmatrix} \Xw(\Psi) & \Xo(\Psi) \\ \Uw(\Psi) & \Uo(\Psi)
	\end{bmatrix} \\ 
	&=
	\begin{bmatrix}
		(I + H\Psi)G & I + H\Psi \\
		\Psi G & \Psi
	\end{bmatrix} \\
	&=
	\begin{bmatrix} I + H\Psi \\ \Psi
	\end{bmatrix}
	\begin{bmatrix} G & I
	\end{bmatrix}
\end{aligned}
\end{equation}

Further, instead of writing a quadratic form as $x^T P x$, I prefer to write it as $Px^{\otimes 2}$, or simply as $Px^2$.
This implies that the quadratic form $P$ is a bilinear mapping within the space $T^0_2(X)$ (that is if $x\in X$), acting on either pairs of members in $X$ or in members in $T^2_0(X)$.
The notation $x^{\otimes 2}$ refers to the tensor square, ie.\ $x\otimes x$, whose $i$-$j$th components are $x^ix^j$.
The application of the quadratic form is $P_{ij}x^ix^j$, using Einstein summation convention, which is equivalent to $x^T P x$ in matrix notation. 
This allows us to write generalized multi-variable polynomails in a notation reminiscent of single variable case, although the notation's meaning is much more general.
Polynomial expansion follows similar rules as for the similar case.
For example:

\begin{equation}
	(\alpha x+\beta y)^2 = \alpha^2 x^2 + 2(\alpha \otimes \beta)(x, y) + \beta^2 y^2
\end{equation}

Here, $\alpha$ and $\beta$ are 1-forms.

We can 

\section{Code}
\begin{verbatim}
import numpy as np
from functools import reduce
import operator
import cvxpy as cp
import scipy as sp

N = 2
n_x = 1
n_u = 1

As = [np.array([[2]]) for k in range(N)]
Bs = [np.array([[1]]) for k in range(N)]
Cs = [np.identity(n_x) for k in range(N)]

Sigma_0 = np.identity(n_x)
Sigma_f = 0.25 * np.identity(n_x)

Qs = [0.5*np.identity(n_x) for k in range(N)]
Rs = [0.5*np.identity(n_u) for k in range(N)]
Rcss = [[0.1*np.identity(n_u) for k in range(N)]]
cbars = [5]

Q = sp.linalg.block_diag(*Qs, np.zeros((n_x,n_x)))
R = sp.linalg.block_diag(*Rs)
Rcs = [sp.linalg.block_diag(*mats) for mats in Rcss]

def transition(matrices,t,tau):
    dimension = matrices[0].shape[0]
    return reduce(operator.matmul, reversed(matrices[tau:t]), np.identity(dimension))

def fill_upper_zeros(rows):
    z = np.zeros(rows[-1][0].shape)
    N = len(rows[-1])

    blocks = [row + [z]*(N - len(row)) for row in rows]
    return np.block(blocks)

def history(Ns, As):
    '''
    Build a history matrix.
    As is the sequence of dynamics operators.
    Ns is the sequence of control or noise matrices.
    '''

    N = len(As)
    histlist = [[transition(As,k,j+1) @ Ns[j] for j in range(k)]
            for k in range(N+1)]
    return fill_upper_zeros(histlist)


H = history(Bs, As)
G = history(Cs, As)

Gamma = np.vstack([transition(As, t, 0) for t in range(len(As)+1)])


psis = [[cp.Variable((n_u, n_x)) for col in range(row+1)] +
        [np.zeros((n_u, n_x))]*(N-row) for row in range(N)]
psi = cp.bmat(psis)

X0 = np.identity(H.shape[0]) + H@psi
Xw = X0 @ G

U0 = psi
Uw = psi @ G

def quad(Mw, M0):
    return Mw@Mw.T + M0@Gamma@Sigma_0@cp.transpose(M0@Gamma) 

xterm = cp.trace(quad(Xw, X0)@Q)
uterm = cp.trace(quad(Uw, U0)@R)
import pdb; pdb.set_trace()
performance_index = cp.Minimize(xterm + uterm)

input_constraints = [(cp.trace(quad(Uw, U0) @ Rc) <= cbar)
        for (Rc, cbar) in zip(Rcs, cbars)]

tcweight = G@G.T + Gamma@Sigma_0@Gamma.T
pns = [np.zeros((n_x,n_x))]*N + [np.identity(n_x)]
PN = np.hstack(pns)
tcquad = PN @ X0
terminal_constraint = cp.PSD(Sigma_f - tcquad@tcweight@cp.transpose(tcquad))

constraints = input_constraints + [terminal_constraint]
problem = cp.Problem(performance_index, constraints)
problem.solve()

F = np.inverse(X0.value) @ psi.value
\end{verbatim}

\end{document}
